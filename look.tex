\documentclass{article}
\usepackage{aeguill,aecompl,tikz,pgfplots,geometry}
\usetikzlibrary{trees}
\usepackage[version=3,arrows=pgf-filled]{mhchem}
\pgfplotsset{compat=newest}
\newcommand{\dirichlet}[2]{\ensuremath{\mathcal{D}\text{#2}\left(#1\right)}}
\newcommand{\Diri}[1]{\dirichlet{#1}{iri}}
\newcommand{\Diun}[1]{\dirichlet{#1}{iun}}
\newcommand{\LogN}[1]{\ensuremath{\mathcal{L}\text{og}\mathcal{N}\left(#1\right)}}
\begin{document}
\section{Dirichlet}
\begin{tikzpicture}
\begin{axis}[clip=false,
             title={$\{0.3 \pm 0.3, 0.5 \pm 0.04, 0.2 \pm 0.01\}$},
             axis lines=none]
\addplot[mark=none] coordinates{
(0.707107,-0.408248)
(0,0.816497)
(-0.707107,-0.408248)
(0.707107,-0.408248)};
\node[below right] at (axis cs:0.707107,-0.408248) {br$_2$};
\node[above]       at (axis cs:0,0.816497) {br$_3$};
\node[below left]  at (axis cs:-0.707107,-0.408248) {br$_1$};
\addplot[mark=x,mark size=3pt] coordinates{(0.,0.)};
%%%
\addplot[red, mark = *, only marks, mark size=0.2pt] table[x = x, y = y]    {data.dat};
\addplot[blue,mark = *, only marks, mark size=0.2pt] table[x = xg, y = yg]  {data.dat};
\addplot[violet,mark = *, only marks, mark size=0.2pt] table[x = xt, y = yt]{data.dat};
\addplot[teal,mark=triangle*,mark size=2pt] coordinates{(0.141421,-0.163299)};
\end{axis}
\end{tikzpicture}\hfill
\begin{tikzpicture}
\begin{axis}[clip=false,
             title={br$_2$ $\leq$ br$_1$ $\leq$ br$_3$},
             axis lines=none]
\addplot[mark=none] coordinates{
(0.707107,-0.408248)
(0,0.816497)
(-0.707107,-0.408248)
(0.707107,-0.408248)};
\node[below right] at (axis cs:0.707107,-0.408248) {br$_2$};
\node[above]       at (axis cs:0,0.816497) {br$_3$};
\node[below left]  at (axis cs:-0.707107,-0.408248) {br$_1$};
\addplot[mark=x,mark size=3pt] coordinates{(0.,0.)};
%%%
\addplot[red,mark = *, only marks, mark size=0.2pt]  table[x = xu, y = yu]{data.dat};
\addplot[teal,mark = *, only marks, mark size=0.2pt] table[x = xo, y = yo]{data.dat};
\end{axis}
\end{tikzpicture}
%%%
\section{Monovariables}
\subsection{Norm}
\begin{tikzpicture}
\begin{axis}[clip=false,
             title={Norm(5,2)},
             axis lines*=left,
             ymin=0,
             xlabel=$x$,
             ybar interval]
\addplot+[hist={data=x,density}] table[x=norm]{data.dat};
\end{axis}
\end{tikzpicture}
\subsection{NorT}
\begin{tikzpicture}
\begin{axis}[clip=false,
             title={NorT(5,2;1.5,10.8)},
             axis lines*=left,
             ymin=0,
             xlabel=$x$,
             ybar interval]
\addplot+[hist={data=x,density}] table[x=nort]{data.dat};
\end{axis}
\end{tikzpicture}
\subsection{LogN}
\begin{tikzpicture}
\begin{axis}[clip=false,
             title={LogN(5e7,0.5)},
             axis lines*=left,
             ymin=0,
             xlabel=$\log_{10}(x)$,
             ybar interval]
\addplot+[hist={data=x,density}] table[x expr = log10(\thisrow{logn})]{data.dat};
\end{axis}
\end{tikzpicture}
\subsection{LogU}
\begin{tikzpicture}
\begin{axis}[clip=false,
             title={LogU(5e7,8e15)},
             axis lines*=left,
             ymin=0,
             xlabel=$\log_{10}(x)$,
             ybar interval]
\addplot+[hist={data=x,density}] table[x expr=log10(\thisrow{logu})]{data.dat};
\end{axis}
\end{tikzpicture}
\subsection{Unif}
\begin{tikzpicture}
\begin{axis}[clip=false,
             title={Unif(5,12)},
             axis lines*=left,
             ymin=0,
             xlabel=$x$,
             ybar interval]
\addplot+[hist={data=x,density}] table[x=unif]{data.dat};
\end{axis}
\end{tikzpicture}

\section{A tree}

A simple tree:
\begin{displaymath}
\renewcommand{\arraystretch}{1.5}
\ce{R ->}\left\{\begin{array}{l}
                \ce{->[0.2-0.6]}\left\{\begin{array}{l}
                  \ce{->[0-1] P1} \\
                  \ce{->[0-1] P2} 
                  \end{array}\right.\\
                 \ce{->[0.3-0.9] P3} 
                \end{array}\right.
\qquad
\ce{R ->}\left\{\begin{array}{l}
                \ce{->[0-1]}\left\{\begin{array}{l}
                  \ce{->[0-1] P1} \\
                  \ce{->[0-1] P2} 
                  \end{array}\right.\\
                 \ce{->[0-1] P3} 
                \end{array}\right.
\end{displaymath}

\begin{center}
\begin{tikzpicture}
\begin{axis}[clip=false,
             axis lines=none]
\addplot[mark=none] coordinates{
(0.707107,-0.408248)
(0,0.816497)
(-0.707107,-0.408248)
(0.707107,-0.408248)};
\node[below right] at (axis cs:0.707107,-0.408248)  {P$_2$};
\node[above]       at (axis cs:0,0.816497)          {P$_3$};
\node[below left]  at (axis cs:-0.707107,-0.408248) {P$_1$};
\addplot[mark=x,mark size=3pt] coordinates{(0.,0.)};
%%%
\addplot[blue,mark = *, only marks, mark size=0.2pt] table[x = xtr, y = ytr]{data.dat};
\addplot[mark=none,red,dashed] coordinates{(0.141421,0.571548) (-0.141421,0.571548)};
\addplot[mark=none,red,dashed] coordinates{(0.424264,0.0816497) (-0.424264,0.0816497)};
\node[left] at (axis cs:-0.141421,0.571548)  {$(0.0,0.2,0.8)$};
\node[right] at (axis cs:0.141421,0.571548)  {$(0.2,0.0,0.8)$};
\node[left] at (axis cs:-0.424264,0.0816497) {$(0.0,0.6,0.4)$};
\node[right] at (axis cs:0.424264,0.0816497) {$(0.6,0.0,0.4)$};
\end{axis}
\end{tikzpicture}
\begin{tikzpicture}
\begin{axis}[clip=false,
             axis lines=none]
\addplot[mark=none] coordinates{
(0.707107,-0.408248)
(0,0.816497)
(-0.707107,-0.408248)
(0.707107,-0.408248)};
\node[below right] at (axis cs:0.707107,-0.408248)  {P$_2$};
\node[above]       at (axis cs:0,0.816497)          {P$_3$};
\node[below left]  at (axis cs:-0.707107,-0.408248) {P$_1$};
\addplot[mark=x,mark size=3pt] coordinates{(0.,0.)};
%%%
\addplot[blue,mark = *, only marks, mark size=0.2pt] table[x = xtrd, y = ytrd]{data.dat};
\end{axis}
\end{tikzpicture}
\end{center}

\begin{tikzpicture}
\matrix{
\begin{axis}[width=5cm,
                ]
\addplot[red,only marks,mark size=0.2pt] table[x=trd1,y=trd2]{data.dat};
\addplot[blue,only marks,mark size=0.2pt] table[x=tr1,y=tr2]{data.dat};
\end{axis}
\\
\begin{axis}[width=5cm,
                ]
\addplot[red,only marks,mark size=0.2pt] table[x=trd1,y=trd3]{data.dat};
\addplot[blue,only marks,mark size=0.2pt] table[x=tr1,y=tr3]{data.dat};
\end{axis}
&
\begin{axis}[width=5cm,
                ]
\addplot[red,only marks,mark size=0.2pt] table[x=trd2,y=trd3]{data.dat};
\addplot[blue,only marks,mark size=0.2pt] table[x=tr2,y=tr3]{data.dat};
\end{axis}
\\
};
\end{tikzpicture}
\end{document}

One reaction:
\begin{displaymath}
\renewcommand{\arraystretch}{1.5}
\ce{CH4+ + e- ->[\LogN{5\,10^{7},0.5}]}\left\{\begin{array}{l}
                        \ce{->[0-1] CH3 + H} \\
                        \left\{\begin{array}{l}
                               \left\{\begin{array}{l}
                                      \ce{->[0-1] ^1CH2 + H2} \\
                                      \ce{->[0-1] ^3CH2 + H2} \\
                                      \end{array}\right.\\
                               \left\{\begin{array}{l}
                                      \ce{->[0-1] ^1CH2 + 2 H} \\
                                      \ce{->[0-1] ^3CH2 + 2 H} \\
                                      \end{array}\right.
                               \end{array}\right. \\
                        \left\{\begin{array}{l}
                                \ce{->[0-1] CH + H2 + H} \\
                                \ce{->[0-1] CH + 3 H} \\
                               \end{array}\right. \\
                        \ce{->[0-1] C + 2 H2}
                        \end{array}\right.
\end{displaymath}
The full pdf is, using \Diri{1,1} for \Diun{2}
\[
\LogN{5\,10^7,0.5} \otimes \Diri{1,1\otimes\Diri{1\otimes\Diun{2},1\otimes\Diun{2}},1\otimes\Diun{2},1}
\]
\begin{tikzpicture}
\matrix{
\begin{axis}[width=5cm,
                ]
\addplot[only marks,mark size=0.5pt] table[x=tr1,y=tr2]{data.dat};
\end{axis}
\\
\begin{axis}[width=5cm,
                ]
\addplot[only marks,mark size=0.5pt] table[x=tr1,y=tr3]{data.dat};
\end{axis}
&
\begin{axis}[width=5cm,
                ]
\addplot[only marks,mark size=0.5pt] table[x=tr2,y=tr3]{data.dat};
\end{axis}
\\
\begin{axis}[width=5cm,
                ]
\addplot[only marks,mark size=0.5pt] table[x=tr1,y=tr4]{data.dat};
\end{axis}
&
\begin{axis}[width=5cm,
                ]
\addplot[only marks,mark size=0.5pt] table[x=tr2,y=tr4]{data.dat};
\end{axis}
&
\begin{axis}[width=5cm,
                ]
\addplot[only marks,mark size=0.5pt] table[x=tr3,y=tr4]{data.dat};
\end{axis}
\\
\begin{axis}[width=5cm,
                ]
\addplot[only marks,mark size=0.5pt] table[x=tr1,y=tr5]{data.dat};
\end{axis}
&
\begin{axis}[width=5cm,
                ]
\addplot[only marks,mark size=0.5pt] table[x=tr2,y=tr5]{data.dat};
\end{axis}
&
\begin{axis}[width=5cm,
                ]
\addplot[only marks,mark size=0.5pt] table[x=tr3,y=tr5]{data.dat};
\end{axis}
&
\begin{axis}[width=5cm,
                ]
\addplot[only marks,mark size=0.5pt] table[x=tr4,y=tr5]{data.dat};
\end{axis}
\\
\begin{axis}[width=5cm,
                ]
\addplot[only marks,mark size=0.5pt] table[x=tr1,y=tr6]{data.dat};
\end{axis}
&
\begin{axis}[width=5cm,
                ]
\addplot[only marks,mark size=0.5pt] table[x=tr2,y=tr6]{data.dat};
\end{axis}
&
\begin{axis}[width=5cm,
                ]
\addplot[only marks,mark size=0.5pt] table[x=tr3,y=tr6]{data.dat};
\end{axis}
&
\begin{axis}[width=5cm,
                ]
\addplot[only marks,mark size=0.5pt] table[x=tr4,y=tr6]{data.dat};
\end{axis}
&
\begin{axis}[width=5cm,
                ]
\addplot[only marks,mark size=0.5pt] table[x=tr5,y=tr6]{data.dat};
\end{axis}
\\
\begin{axis}[width=5cm,
                ]
\addplot[only marks,mark size=0.5pt] table[x=tr1,y=tr7]{data.dat};
\end{axis}
&
\begin{axis}[width=5cm,
                ]
\addplot[only marks,mark size=0.5pt] table[x=tr2,y=tr7]{data.dat};
\end{axis}
&
\begin{axis}[width=5cm,
                ]
\addplot[only marks,mark size=0.5pt] table[x=tr3,y=tr7]{data.dat};
\end{axis}
&
\begin{axis}[width=5cm,
                ]
\addplot[only marks,mark size=0.5pt] table[x=tr4,y=tr7]{data.dat};
\end{axis}
&
\begin{axis}[width=5cm,
                ]
\addplot[only marks,mark size=0.5pt] table[x=tr5,y=tr7]{data.dat};
\end{axis}
&
\begin{axis}[width=5cm,
                ]
\addplot[only marks,mark size=0.5pt] table[x=tr6,y=tr7]{data.dat};
\end{axis}
\\
\begin{axis}[width=5cm,
                ]
\addplot[only marks,mark size=0.5pt] table[x=tr1,y=tr8]{data.dat};
\end{axis}
&
\begin{axis}[width=5cm,
                ]
\addplot[only marks,mark size=0.5pt] table[x=tr2,y=tr8]{data.dat};
\end{axis}
&
\begin{axis}[width=5cm,
                ]
\addplot[only marks,mark size=0.5pt] table[x=tr3,y=tr8]{data.dat};
\end{axis}
&
\begin{axis}[width=5cm,
                ]
\addplot[only marks,mark size=0.5pt] table[x=tr4,y=tr8]{data.dat};
\end{axis}
&
\begin{axis}[width=5cm,
                ]
\addplot[only marks,mark size=0.5pt] table[x=tr5,y=tr8]{data.dat};
\end{axis}
&
\begin{axis}[width=5cm,
                ]
\addplot[only marks,mark size=0.5pt] table[x=tr6,y=tr8]{data.dat};
\end{axis}
&
\begin{axis}[width=5cm,
                ]
\addplot[only marks,mark size=0.5pt] table[x=tr7,y=tr8]{data.dat};
\end{axis}
\\
};
\end{tikzpicture} 
\end{document}
